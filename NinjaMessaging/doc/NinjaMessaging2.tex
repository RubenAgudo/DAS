\documentclass[]{article}
\usepackage[spanish]{babel}

%opening
\title{NinjaMessaging 2}
\author{Rub\'en Agudo Santos}

\begin{document}

\maketitle

\begin{abstract}
En la segunda versi\'on de NinjaMessaging se ha trabajado con notificaciones push, bases de datos remotas, preferencias,
widgets y mapas. En las subsecuentes secciones se explicar\'a detalladamente que se ha hecho.
\end{abstract}

\section{Funcionalidades obligatorias}
Aqu\'i se presentar\'an las funcionalidades m\'inimas para obtener un 5.
\subsection{Notificaciones push}
Las notificaciones push solo van a funcionar cuando desde la pantalla principal se pinche en el ``+" verde en la esquina
superior derecha. Al pinchar ah\'i se mostrar\'an todos los n\'umeros IMEI registrados. Es decir, no existe el concepto de amistad,
m\'as bien podr\'ia considerarse una sala de chat m\'ovil, en la que una vez te has registrado, ya no hay manera de darse de baja.

Una vez seleccionada una conversaci\'on (tambi\'en aparece tu propio dispositivo por motivos de testeo) puedes escribir lo que
sea, y llegar\'a una notificaci\'on push al dispositivo que est\'es escribiendo. La notificaci\'on, si es de texto plano, al 
pinchar sobre ella te llevar\'a a la pantalla principal, y NO se mostrar\'a el texto que te ha escrito, ese texto solo ser\'a
visible a trav\'es de la propia notificaci\'on.

Tambi\'en se ha usado la clase SharedPreferences para guardar el \emph{regid} actual. Idealmente, habr\'ia que crear un mecanismo 
en el cual si Google cambia el \emph{regid} a tu dispositivo pueda actualizarse autom\'aticamente. Lo mas sencillo ser\'ia cambiar
la versi\'on de la aplicaci\'on, y en la base de datos remota, en la tabla registro, guardar como clave primaria el usuario y
la versi\'on de la aplicaci\'on.

\subsection{Bases de datos remotas}
La base de datos remota consiste \'unicamente de una tabla que se compone de dos campos, usuario, que es el IMEI del dispositivo
m\'ovil y el regid, que es el registration id proporcionado por google.

Esa tabla se utiliza para poder enviar mensajes push a trav\'es de GCM y curl

Los datos que se obtienen son unicamente los IMEI, que se muestran en la pantalla de Contacts. Es una simple \emph{SELECT}
para mostrar todos los usuarios del sistema.

La \emph{INSERT} que se realiza, es para guardar los usuarios registrados, con su numero IMEI + regid.

\subsection{Widgets}
Se ha creado un widget muy sencillo, con una actividad de configuraci\'on, que al lanzarlo, debes introducir el nombre del usuario
del que quieres ver el \'ultimo mensaje.

Una vez has puesto el usuario, se mostrar\'a autom\'aticamente el \'ultimo mensaje de esa conversaci\'on, ya sea tuyo o suyo. El
Widget contiene un bot\'on que te permite actualizar el widget y comprobar si el \'ultimo mensaje ha cambiado.

\section{Funcionalidades optativas}
Aqu\'i se presentar\'an las funcionalidades que se utilizan para subir la nota del proyecto.

\subsection{Preferencias}
En el documento donde se especificaban los requisitos m\'inimos, se planteaba el uso de preferencias que tuviesen alg\'un tipo
de efecto en el funcionamiento del programa.

En mi caso se ha decidido que las preferencias tengan un \emph{checkbox} que inicia o detiene el servicio que notifica cada
15 segundos de que tienes un nuevo mensaje. Como el cambio ha de hacerse de manera instant\'anea y el \emph{polling} no
merece la pena, se ha decidido implementar la interfaz \emph{OnSharedPreferenceChangeListener} que contiene el m\'etodo
\emph{onSharedPreferenceChanged(SharedPreferences sp, String key)} en el que te indica que clave ha cambiado, para que se
pueda actuar en consecuencia.

Tambi\'en es importante comentar que se ha utilizado un \emph{PreferenceFragment} embebido dentro de una \emph{Activity}
normal, tal y como recomienda Google.

Para acceder a las preferencias, se hace desde la pantalla principal, dando al men\'u de opciones, y pinchando en preferencias.

\subsection{Mapas}
Tambi\'en se han implementado los mapas de google, para enviar a un contacto nuestra \'ultima posici\'on conocida. Para ello se ha
adaptado el laboratorio 7. 

Para conseguir el efecto de enviar la localizaci\'on, lo que se ha hecho es que a trav\'es de GCM
se env\'ian distintos datos, entre ellos un flag \emph{isMap} en el que si es una localizaci\'on, el intent de la 
notificaci\'on te lleva a la actividad del mapa, centrandose en la posici\'on enviada y a\~nadiendo un marker para darle
mayor vistosidad al conjunto. 

La \'unica manera de entrar al mapa es recibiendo una notificaci\'on de mapa.

\end{document}
