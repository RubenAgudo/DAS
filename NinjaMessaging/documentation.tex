\documentclass{article}
\usepackage{fontspec}
\usepackage[spanish]{babel}
\usepackage{listings}
\usepackage{parskip}
\usepackage[hidelinks]{hyperref}
\setmainfont{Helvetica Neue}

\setlength{\parindent}{15pt}

\begin{document}
\author{Rub\'{e}n Agudo Santos}
\title{NinjaMessaging}
\date{\today}
\maketitle

\begin{abstract}
En este proyecto individual se ha decidido realizar una aplicaci\'{o}n de 
mensajer\'{i}a. Obviamente no permite ning\'{u}n tipo de mensajer\'{i}a real,
solo la simulaci\'{o}n con dos usuarios predefinidos. Todo el c\'odigo fuente
est\'a disponible en \url{https://github.com/RubenAgudo/DAS/tree/master/NinjaMessaging} 
licenciado bajo licencia GPLv3.
\end{abstract}

\tableofcontents

\section{Funcionalidades obligatorias}

\subsection{Localizaci\'{o}n}
Este c\'{o}digo est\'{a} contenido en la clase \emph{ChatActivity.java} y empieza cuando el
usuario pincha sobre el icono de localizaci\'{o}n en el Action Bar. 

Entonces se recoge el LocationManager y todos los providers. Nos da igual de donde
provenga la \'{u}ltima localizaci\'{o}n, siempre que la haya.

Entonces obtenemos la direcci\'{o}n f\'{i}sica, si tenemos conexi\'{o}n a internet,
si no, nos devuelve la latitud y la longitud. Finalmente se guarda el mensaje y se
refresca el ListFragment que contiene los mensajes de ese usuario.

\subsection{Bases de datos SQLite}
Para gestionar las bases de datos lo mejor es crear un gestor de bases de datos, 
que en este caso ser\'{a} una MAE (M\'{a}quina abstracta de ejecuci\'{o}n) que solo
tiene una \'{u}nica instancia. De este modo, todas las clases acceden a la misma instancia
y realizan las operaciones necesarias.

Se han creado los m\'{e}todos necesarios para obtener los mensajes de un usuario, todos
los usuarios, los chats recientes, para a\~{n}adir un mensaje a la conversaci\'{o}n de un
usuario.

Tambi\'{e}n se han creado unas funciones ``miscelaneas´´ menos relacionadas con los propios
chats y usuarios. Esas son las de exportar el chat de un usuario a la SD, obtener un mensaje
al azar de un usuario bas\'{a}ndose en unos Strings predefinidos en la propia clase para 
simular una conversaci\'{o}n y la posibilidad de actualizar la informaci\'{o}n de los 
usuarios.

Como curiosidad, decir que he utilizado los m\'etodos insert, update, select propios de
Android, ya que al utilizar expresiones del tipo: 


\begin{lstlisting}[language=java, numbers=left]
db.update("Usuarios", values, "nombreUsuario=?", new String[]{detallesDe});
\end{lstlisting}

Evitamos que haya inyecciones SQL, es menos propenso a fallos que escribir una sentencia
SQL pura con todas las comillas simples, dobles, concatenaciones de Strings etc. Y adem\'as,
al ser algo propio de Android, est\'a considerado como un buen h\'abito de programaci\'on.

\subsection{Notificaciones}
Cada vez que abres la aplicaci\'on, cada 15 segundos (no configurables), la aplicaci\'on
env\'ia una notificaci\'on con un mensaje. Cuando pinchas sobre la notificaci\'on, te
lleva al chat de ese usuario, y la notificaci\'on desaparece.

El c\'odigo puede consultarse en \emph{NotificationService.java}.

\subsection{Fragments}
Se han utilizado principalmente ListFragments, que muestran o bien los chats recientes, o 
los propios mensajes.

Si estamos en la ventana principal, y ponemos en horizontal el dispositivo, en la misma
pantalla se mostrar\'an dos fragments: El que contiene la lista de chats recientes y el
fragment del propio chat.

Se ha controlado que si no hay ning\'un chat seleccionado, que muestre un aviso de que hay
que seleccionar una conversaci\'on. 

\subsection{Men\'{u}s y Action Bar}
A lo largo de la aplicaci\'on, se utilizan en distintos sitios men\'us y acciones de la
Action Bar.

Por ejemplo, cuando estas en la vista de Chat en modo retrato, es decir, hablando con un
usuario, tenemos distintas acciones y opciones.

\begin{itemize}
	\item \textbf{Action Bar, icono de localizaci\'on:} Como he descrito con anterioridad,
	env\'ia la \'ultima posici\'on conocida transformada en direcci\'on f\'isica si se
	dispone de conexi\'on a internet, si no la latitud y la longitud.
	\item \textbf{Action Bar, icono de disquete:} Exporta la conversaci\'on actual a la SD, a un 
	fichero de texto plano.
	\item \textbf{Men\'u, Ver detalles de usuario:} Abre una nueva actividad donde se muestran
	los datos de ese usuario.
\end{itemize}

Para consultar ese c\'odigo, esta disponible en el archivo \emph{ChatActivity.java} para ver
las llamadas y como se tratan los clicks en el men\'u, y en \emph{/menu/chat\_activity.xml}
para ver como est\'an hechos los men\'us.

\subsection{Comentarios en el c\'odigo}
Siguiendo los principios de clean code de Robert C. Martin no he puesto
demasiados comentarios, solo los necesarios en lo que he considerado oportuno porque no 
lo hemos dado en clase o porque el c\'odigo no podr\'ia entenderse todo lo bien que deber\'ia.

Aun as\'i, pr\'acticamente todos los m\'etodos disponen de \emph{javadoc}, para poder
consultar que hace cada funci\'on.

\section{Funcionalidades extra}
\subsection{Servicio}
He creado un servicio que dispone de un \emph{Handler} y de un \emph{Runnable} para que
cada 15 segundos cree una notificaci\'on.

Para crear la notificaci\'on, obtiene de la Base de Datos un chat al azar, que devuelve un
usuario y un texto de mensaje y posteriormente se notifica.

El servicio tambi\'en se lanza cada vez que se lanza la aplicaci\'on, para que no sea
necesario apagar y encender el m\'ovil. Esta opci\'on en un futuro ser\'a configurable.

Lo que se hace en la notificaci\'on ha sido previamente explicado, en la secci\'on de 
notificaciones.

En c\'odigo relativo al servicio puede consultarse en \emph{NotificationService.java}.

\subsection{Receiver}
El Receiver, est\'a estrechamente relacionado con el servicio y con las notificaciones,
ya que lo he creado para que al completar el arranque se lance el servicio que comienza las
notificaciones cada 15 segundos.

Para ello he creado el propio Receiver extendiendo de BroadCastReceiver y en el manifest.xml
le he dado permisos para ejecutar cosas en el BOOT COMPLETED mediante el c\'odigo:

\begin{lstlisting}[language=XML, numbers=left]
<uses-permission android:name="android.permission.RECEIVE_BOOT_COMPLETED" />
\end{lstlisting}

\subsection{Internacionalizaci\'on y globalizaci\'on}
Para la globalizaci\'on e internacionalizaci\'on, he decidido utilizar el m\'etodo
recomendado por Android, que es replicar el fichero \emph{strings.xml} dentro de otras
carpetas que contienen el c\'odigo de idioma correspondiente, es decir, 
\emph{values-en, values-de} etc.

De este modo, el sistema obtendr\'a el fichero adecuado para el idioma en el que est\'e el
dispositivo. Tambi\'en es verdad que hay algunos idiomas que Android no soporta por defecto, y
que ser\'ia preferible dar al usuario la opci\'on de forzar el idioma, pero eso ser\'a para
futuras actualizaciones.

\subsection{Uso de ficheros}
la \'unica manera de dar sentido al uso de ficheros dentro de mi aplicaci\'on era dar
al usuario la oportunidad de exportar los mensajes que ha mantenido con un usuario a la SD
en texto plano.

Esta funcionalidad est\'a disponible desde la pantalla de chat en el Action Bar, que tiene
un icono de disquete.

\subsection{GeoCoder}
Como no ten\'ia mucho sentido utilizar un LocationListener para ``seguir´´ a un usuario
lo que se me ocurri\'o, fue que utilizando la \'ultima posici\'on conocida del m\'ovil, 
obtener la direcci\'on f\'isica.

Para conseguir esto se necesitan dos cosas: Conexi\'on a internet y la clase GeoCoder de
Android. Como el servicio al que consulta el GeoCoder no est\'a en el n\'ucleo de Android,
requiere de conexi\'on a internet.

Realmente lo que se hace es lo conocido como ``Reverse Geocoding´´. La precisi\'on de
este sistema seg\'un los desarrolladores de Android es variable, ya que a veces puede darte
una direcci\'on muy precisa, y otras veces simplemente la ciudad.

El c\'odigo para el Reverse GeoCoding puede consultarse en \emph{ChatActivity.java}.

\subsection{Alert Dialog con layout personalizado}
Tambi\'en he implementado la posibilidad de actualizar los datos personales de los usuarios
mediante un Alert Dialog con un layout personalizado.

Para ello, he creado un layout, y cuando se pincha en el bot\'on del Action Bar dentro
de la actividad DetallesUsuario.java, llama al m\'etodo editarUsuario, y ah\'i, se obtiene
el LayoutInflater, se obtiene una View despu\'es de inflar el layout y solo se pone un bot\'on
de aceptar.

N\'otese que al pinchar no carga los datos actuales, ya que eso no est\'a programado, y ser\'ia
l\'ogico que lo hiciera, pero eso se realizar\'a en futuras actualizaciones.
\end{document}